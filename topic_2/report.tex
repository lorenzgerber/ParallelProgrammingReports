\documentclass[a4paper,11pt,twoside]{article}
%\documentclass[a4paper,11pt,twoside,se]{article}

\usepackage{UmUStudentReport}
\usepackage{verbatim}   % Multi-line comments using \begin{comment}
\usepackage{courier}    % Nicer fonts are used. (not necessary)
\usepackage{pslatex}    % Also nicer fonts. (not necessary)
\usepackage[pdftex]{graphicx}   % allows including pdf figures
\usepackage{listings}
\usepackage{pgf-umlcd}
\usepackage{blindtext}
\usepackage{enumitem}
%\usepackage{lmodern}   % Optional fonts. (not necessary)
%\usepackage{tabularx}
%\usepackage{microtype} % Provides some typographic improvements over default settings
%\usepackage{placeins}  % For aligning images with \FloatBarrier
%\usepackage{booktabs}  % For nice-looking tables
%\usepackage{titlesec}  % More granular control of sections.

% DOCUMENT INFO
% =============
\department{Department of Computing Science}
\coursename{Parallel Programming 7.5 p}
\coursecode{5DV152}
\title{Exercises, Chapter/Topic 1}
\author{Lorenz Gerber ({\tt{dv15lgr@cs.umu.se}} {\tt{lozger03@student.umu.se}})}
\date{2017-01-26}
%\revisiondate{2016-01-18}
\instructor{Lars Karlsson / Mikael Ränner}


% DOCUMENT SETTINGS
% =================
\bibliographystyle{plain}
%\bibliographystyle{ieee}
\pagestyle{fancy}
\raggedbottom
\setcounter{secnumdepth}{2}
\setcounter{tocdepth}{2}
%\graphicspath{{images/}}   %Path for images

\usepackage{float}
\floatstyle{ruled}
\newfloat{listing}{thp}{lop}
\floatname{listing}{Listing}



% DEFINES
% =======
%\newcommand{\mycommand}{<latex code>}

% DOCUMENT
% ========
\begin{document}
\lstset{language=C}
\maketitle
\thispagestyle{empty}
\newpage
\tableofcontents
\thispagestyle{empty}
\newpage

\clearpage
\pagenumbering{arabic}

\section{Introduction}
This report is part of the mandatory coursework. It describes the solutions for several chosen exercises from the course book \cite{pacheco2011}.
\section{2.4 - Counting pages}
$2^{20}$
\section{2.8 - Hardware multithreading and caches}
Caching operates on whole cache lines. Hence if the chance that another process/thread changes something in a specific cache line increases with cache size and number of processes/threads. The specific situation that can happen is called `false sharing': When one process changes data in a cache line, there is no possibilty to check or know for another process which data exactly was changed. It can very well be the case that to the current process unrelated data was changed and a cache reload would not be needed. However, cache has to be reloaded. 
\section{2.10 - Communication overhead}
The calculations can be split in two parts: the instructions and the communication. For both a) and b) the instructions will take the same time:\\
\verb+instructions / cores / instructions_per_second+\\
Hence $10^{12} \div 1000 \div 10^{6} = 1000 sec$. For communications, the time is calculated as\\
\verb+messages_to_send * time_per_message+\\
while messages to send is described as $10^{9}(p-1)$. 
\begin{enumerate}[label={\alph*)}]
\item if sending one message takes $10^{-9}$ seconds, the communication with 1000 cores will take: $10^{9}(1000-1) \times 10^{-9} = 99'900 sec$, together with the actual calculation, $100'900 sec = 28.03h$.
\item if sending one message takes $^{-3}$ seconds , the communication with 1000 cores will take: $10^{9}(1000-1) \times 10^{-3} = 99'900'000'000 sec$, together with the actual calculation, $99'900'001'000 sec = 3'167.8 years$.
\end{enumerate}


    
\section{2.16 - Speedup efficiency}
\section{2.19 - Scalability}
\section{2.20 - Linear speedup and strong scalability}
\section{2.23 - Alternative algorithm for computing histogram}
\addcontentsline{toc}{section}{\refname}
\bibliography{references}

\end{document}
