\documentclass[a4paper,11pt,twoside]{article}
%\documentclass[a4paper,11pt,twoside,se]{article}

\usepackage{UmUStudentReport}
\usepackage{verbatim}   % Multi-line comments using \begin{comment}
\usepackage{courier}    % Nicer fonts are used. (not necessary)
\usepackage{pslatex}    % Also nicer fonts. (not necessary)
\usepackage[pdftex]{graphicx}   % allows including pdf figures
\usepackage{listings}
\usepackage{pgf-umlcd}
\usepackage{blindtext}
\usepackage{enumitem}
\usepackage{amsmath}
\usepackage{amssymb}
%\usepackage{lmodern}   % Optional fonts. (not necessary)
%\usepackage{tabularx}
%\usepackage{microtype} % Provides some typographic improvements over default settings
%\usepackage{placeins}  % For aligning images with \FloatBarrier
%\usepackage{booktabs}  % For nice-looking tables
%\usepackage{titlesec}  % More granular control of sections.

% DOCUMENT INFO
% =============
\department{Department of Computing Science}
\coursename{Parallel Programming 7.5 p}
\coursecode{5DV152}
\title{Final Programming Assignment\\ MPI Game of Life}
\author{Lorenz Gerber ({\tt{dv15lgr@cs.umu.se}} {\tt{lozger03@student.umu.se}})}
\date{2017-03-22}
%\revisiondate{2016-01-18}
\instructor{Lars Karlsson / Mikael Ränner}


% DOCUMENT SETTINGS
% =================
\bibliographystyle{plain}
%\bibliographystyle{ieee}
\pagestyle{fancy}
\raggedbottom
\setcounter{secnumdepth}{2}
\setcounter{tocdepth}{2}
%\graphicspath{{images/}}   %Path for images

\usepackage{float}
\floatstyle{ruled}
\newfloat{listing}{thp}{lop}
\floatname{listing}{Listing}



% DEFINES
% =======
%\newcommand{\mycommand}{<latex code>}

% DOCUMENT
% ========
\begin{document}
\lstset{language=C}
\maketitle
\thispagestyle{empty}
\newpage
\tableofcontents
\thispagestyle{empty}
\newpage

\clearpage
\pagenumbering{arabic}

\section{Introduction}
Here it was the aim to design and implement conway's game of life that can be run on a High Performance Computing system using the MPI library. 
\section{Conway's Game of Life}
Conway
simplify mathematical models from von Neuman that were about building copies of themselves. Published 1970, the surprising property of the program was that it represents a turing complete system despite using only very simple rules. It represents the start of research into `cellular automata'.

Conway's game of life is represented by a regular grid of limited size consisting of individual cells that can be either alive or death. The game of life proceeds in rounds: In each round the state of each cell is evaluated individually and then updated to an eventual new state. Then the next round starts. The most important feature of this system are the rules applied for assessing and updating the state of each individual cell: Based on the eight neighbouring cells, it is determined if a cells environment is life promoting or overcrowded. Hence, the variable to determine and evaluate is the number of neighbouring cells alive.       

\section{Implementation}
For implementing an MPI version of Conway's Game of Life, `Foster's Methodology' as reproduced in the course litterature \cite[p. 66]{pacheco2011} was applied. Here it was required to implement the program for a distributed memeory MPI system. This has significant influence on how Foster's Methodology is evaluated.

First the problem should be defined.

\verbatim{Pseudo Code Serial}
# set up game
for i in x1 ->  xk {
  for j in y1 -> yk {
    set start_value xi, yj
  }
}

for num_generations {
  for i in x1 -> xk {
    for j in y1 -> yk {
      temp_field = determine_status(x,y)
    }
  }
  

}


  


\begin{enumerate}
\item Partitioning\\
To decide on partitioning, some desci


\item Communication
\item Agglomeration / Aggregation
\item Mapping
\end{enumerate}


\section{Benchmarking}



\addcontentsline{toc}{section}{\refname}
\bibliography{references}

\appendix
\section{C Source Code TSP Condways Game of Life}{\label{app:gol}}


\end{document}
