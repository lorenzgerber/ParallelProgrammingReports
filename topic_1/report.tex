\documentclass[a4paper,11pt,twoside]{article}
%\documentclass[a4paper,11pt,twoside,se]{article}

\usepackage{UmUStudentReport}
\usepackage{verbatim}   % Multi-line comments using \begin{comment}
\usepackage{courier}    % Nicer fonts are used. (not necessary)
\usepackage{pslatex}    % Also nicer fonts. (not necessary)
\usepackage[pdftex]{graphicx}   % allows including pdf figures
\usepackage{listings}
\usepackage{pgf-umlcd}
\usepackage{blindtext}
\usepackage{enumitem}
\usepackage{amsfonts}
%\usepackage{lmodern}   % Optional fonts. (not necessary)
%\usepackage{tabularx}
%\usepackage{microtype} % Provides some typographic improvements over default settings
%\usepackage{placeins}  % For aligning images with \FloatBarrier
%\usepackage{booktabs}  % For nice-looking tables
%\usepackage{titlesec}  % More granular control of sections.

% DOCUMENT INFO
% =============
\department{Department of Computing Science}
\coursename{Parallel Programming 7.5 p}
\coursecode{5DV152}
\title{Exercises, Chapter/Topic 1}
\author{Lorenz Gerber ({\tt{dv15lgr@cs.umu.se}} {\tt{lozger03@student.umu.se}})}
\date{2017-01-26}
%\revisiondate{2016-01-18}
\instructor{Lars Karlsson / Mikael Ränner}


% DOCUMENT SETTINGS
% =================
\bibliographystyle{plain}
%\bibliographystyle{ieee}
\pagestyle{fancy}
\raggedbottom
\setcounter{secnumdepth}{2}
\setcounter{tocdepth}{2}
%\graphicspath{{images/}}   %Path for images

\usepackage{float}
\floatstyle{ruled}
\newfloat{listing}{thp}{lop}
\floatname{listing}{Listing}



% DEFINES
% =======
%\newcommand{\mycommand}{<latex code>}

% DOCUMENT
% ========
\begin{document}
\lstset{language=C}
\maketitle
\thispagestyle{empty}
\newpage
\tableofcontents
\thispagestyle{empty}
\newpage

\clearpage
\pagenumbering{arabic}

\section{Introduction}
This report is part of the mandatory coursework. It describes the solutions for several chosen exercises from the course book \cite{pacheco2011}.
\section{1.1 - Formulas for block partitioning}
The overwhelming idea is to load balance \verb+p+ number of cores with \verb+n+ tasks. Here, we use two functions to obtain block partitioning using a \verb+for loop+ :
\begin{verbatim}
for (my_i = my_first_i; my_i < my_last_i; my_i++)
\end{verbatim}

The functions \verb+my_first_i+ and \verb+my_last_i+ are used to set the limits in the loop. Besides \verb+n, i and p+ we also need an index for the actual core: \verb+k+. It is understood that indicies \verb+i+ and \verb+k+ start at \verb+0+. The book text hints to start with the case when \verb+n+ is evenly divisible by \verb+p+:
\begin{verbatim}
my_first_i = k * n / p
my_last_i = (k + 1) * n / p
\end{verbatim}
Testing this expression for \verb!n = 10, p = 5, k = {0, 1, ..., 4}! seems to be correct. Now when  \verb+n+ is not even divisible by \verb+p+, one has to distribute the \verb+n mod p+ tasks for example on the first \verb+n mod p+ cores:
\begin{verbatim}
my_first_i = k * n / p + ( k < n mod p ? k : n mod p )
my_last_i = (k + 1) * n / p + (k + 1 < n mod p ? k + 1 : n mod p)  
\end{verbatim}
Testing this expression for \verb!n = 9, p = 5, k = {0, 1, ..., 4}! gives the correct results.
  
\section{1.2 - Modify 1.1 with non-uniform costs}
In my opinion, this question can be understood in several different ways:

\subsection{First attempt}
Let there be 10 tasks (\verb+n+) that take \verb+2, 4, ..., 20 ms+ to finish, and there are 3 cores (\verb+p+). So one could try to distribute the tasks onto the cores that the standard deviation of total runtimes for all cores is as little as possible. A possible solution for the above described actual setting would be:\\ $i_{9}$ to $k_{0}$, $i_{8}$ to $k_{1}$, $i_{7}$ to $k_{2}$. Then,  $i_{0}$ to $k_{0}$, $i_{1}$ to $k_{1}$, $i_{2}$ to $k_{2}$. And $i_{6}$ to $k_{0}$, $i_{5}$ to $k_{1}$, $i_{4}$ to $k_{2}$. Finally, $i_{3}$ to $k_{0}$. This somehow works, makes sense and I guess it would be possible to devise a general formula for this.\\
However, it does not adhere to the actual question: \textit{How would you change your answer to the preceding question if ...}. And the preceding question is: \textit{Devise formulas for the functions that calculate `my\_first\_i' and `my\_last\_i' in the global sum example.} Hence, to make formulas for first \verb+i+ and last \verb+i+ the tasks should be assigned to the cores in a consecutive sequence. Therefore, second attempt:

\subsection{Second attempt}
If the tasks shall be processed in order, one can express the total elapsed time for a sequential solution as $N = n^{2} + n$ (where \verb+n+ is the number of tasks) and simlar at any intermediate point as $I = i^{2} + i$ (where \verb+i+ is the index of \verb+n+). The average time that every core should be busy, can be expressed as the total time elapsed divided by number of cores: $M\frac{N}{p}$. Now one can devise an equality for intermediate \verb+i+ values: $i^{2}+i = k*M$ where \verb+k+ is the core index. This is a quadratic equation that can be solved for \verb+i+ in $\mathbb{N}$ (hence, results should be rounded to the closest integer value). The obtained values are the breaks between a sequence of tasks. 
\begin{itemize}
\item For \verb+my_last_i+, the above expression is modified to  $i^{2}+i=(k + 1)M$
\item $my\_last\_i(i_{q})$ for $q=0$ is by definition \verb+0+, and for $i_{q}, q > 0$ it is $my\_last\_i(i_{q-1}) + 1$. 
\end{itemize}
 
For practical usage, the above formula has to be reformulated to have a single occurence of \verb+i+ on one side of the equation:
\begin{equation}
i=\frac{\sqrt{4kn^{2}+4kn+4n^{2}+4n+p}\sqrt{p}-p}{2p}
\end{equation}
However, this solution seems to me not very meaningful as it produces longer run-times than the solution in the first attempt. Therefore I tried to look again more detailed into how the question is formulated and came up with:

\subsection{Third attempt}
The question states \verb+i = k+ requires \verb!k + 1! times as much time than call \verb+i = 0+. Somehow this does not make sense. Because in this formula, to define the time, there need to be equal indices k as indices i. That would mean that the problem is by definition embarrasingly parallel.

From the wording of the question in the second part \textit{the first \textbf{call} takes..., the second \textbf{call} requires...} etc., I conclude that the author means that every time the algorithm is executed it will take inherently (in the current case) 2 milliseconds longer to finish. Maybe it should not be \verb+k+, as index for cores, but just an arbitrary index to construct a number series. This brings me to the fourth and last idea:

\subsection{Fourth attempt}
If every call to the algorithm takes inherently longer, the task's \verb+i+ are actually independent of the time it takes to run and one can not start executing the `long running' \verb+i+'s as described in attempt 1 and 2. In this case, it's obvious that the solution for 1.1 is also valid here. 

\section{1.3 - Tree-structured global sum}
The aim was to write pseudo code that calculates the tree structured global sum described on page 5. The book hints to use a variable called \verb+divisor+ that is initialized with the value \verb+2+ and another variable called \verb+core_difference+ that is initialized with the value \verb+1+. It was proposed that \verb+divisor+ is doubled in each iteration and that \verb+n mod divisor+ is used to determine \verb+send = 0+ and \verb+receive = 1+. From figure 1.1 in the book, one can see that this rule works. Note especially that core \verb+k = 0+ will read in each iteration as \verb+0 mod x = 0+ is true for any \verb+x+. Further, the book proposes that \verb+core_diff+, when doubled in each iteration, can be used to describe the difference in value between a core pair that is involved in a `send/receive' operation. The correctness of this can also be verified in figure 1.1 of the book.

Assuming that \verb+k+ is the core index, this allows already to write the main part of the algorithm:
\begin{verbatim}
if( k % divisor == 0 )
    receive and add from k + core_diff
else
    send to k - core_diff
    break
\end{verbatim}
From figure 1.1, it can also be seen that \verb+send+ is the last operation a core does. Afterwards, it can terminate. This is solved here with a break statement after send. Further, there need to be some control structure to know when to finish. Here it was decided to use a \verb+while+ loop with the comparison \verb+core_diff < p+, where \verb+p+ is the number of cores in the system. In each iteration \verb+divisor+ and \verb+core_diff+ are doubled. Moreover, before exit, the core \verb+k = 0+ should return the result. Below is the complete code listing that incorporates the the described features:

\begin{verbatim}
divisor = 2
core_diff = 1

while(core_diff < p){
    if( k % divisor == 0 )
        receive and add from k + core_diff
    else 
        send to k - core_diff
        break

    divisor * 2
    core_diff * 2
}

if( k == 0 )
    return final result
\end{verbatim}

\section{1.4 - Alternative algorithm for 1.3}
Here the aim was to modify the pseudo code from 1.3 to use bit-wise operaters, basically to obtain \verb+k+ indices for \verb+send+ and \verb+receive+. In the book, the idea is visually described using a table (page 13, exercise 1.4). The bitwise \verb+Xor+ operator is applied to a binary \verb+bitmask+ with the initial value $001_{2}$ and the binary value of each \verb+k+. This will flip the last bit in every \verb+k+ value, resulting in two \verb+k+'s exchanging their initial value with eachother. This can be exploited to use the core \verb+k+ whose value becomes lower by the bitshift as the \verb+sender+ and the other one in the `flipping pair' as \verb+receiver+. After the first iteration, the \verb+bitmask+ is bit-shifted to the left, hence $001_{2}$ becomes $010_{2}$. Now follows the next iteration round with applying Xor to all remaining \verb+k+'s.      

Here the value of the bitmask itself can be used in a \verb+while+ loop to determine the end of the iteration using the comparison: \verb+bitmask =< p+. Also in this version, after the while loop the core \verb+k = 0+ needs to return the result.

\begin{verbatim}
bitmask = 1

while ( bitmask =< p ){
    if(k bitwiseXor bitmask > k)
        receive and add from k bitwiseXor bitmask
    else
        send to k bitwiseXor bitmask
        break

    bitwiseLeft bitmask
}

if( k == 0)
    return final result

\end{verbatim}



\section{1.5 - Generalization of 1.3 and 1.4}
Here it was required to modify 1.3 and 1.4 to a more generalized form that could also handle the case when \verb+p+ is not a power of two value.
Basically the same code could be maintained with just an additional \verb+if+ control sequence around the `receive' statement that handles the case when there is no neighbour in the graph that could send the result. 


\subsection{Generalized form or 1.3}
\begin{verbatim}
divisor = 2
core_diff = 1

while(core_diff < p){
    if(k % divisor == 0)
        if (k + core_diff < p)  
            receive and add from k + core_diff
    else 
        send to k - core_diff
        break

    divisor * 2
    core_diff * 2
}

if( k == 0 )
    return final result
\end{verbatim}

\subsection{Generalized form of 1.4} 
\begin{verbatim}
bitmask = 1

while ( bitmask =< p ){
    if(k bitwiseXor bitmask > k)
        if (k bitwiseXor bitmask < p)
            receive and add from k bitwiseXor bitmask
    else
        send to k bitwiseXor bitmask
        break

    bitwiseLeft bitmask
}

if( k == 0)
    return final result
\end{verbatim}

\section{1.6 - Cost anlaysis of global sum algorithms}
The number of receives and additions for core 0 is in the original `pseudo-code global sum' \verb+p-1+ and in the tree-structured global sum \verb+ceiling(log2(p))+.

\addcontentsline{toc}{section}{\refname}
\bibliography{references}

\end{document}
