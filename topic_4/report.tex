\documentclass[a4paper,11pt,twoside]{article}
%\documentclass[a4paper,11pt,twoside,se]{article}

\usepackage{UmUStudentReport}
\usepackage{verbatim}   % Multi-line comments using \begin{comment}
\usepackage{courier}    % Nicer fonts are used. (not necessary)
\usepackage{pslatex}    % Also nicer fonts. (not necessary)
\usepackage[pdftex]{graphicx}   % allows including pdf figures
\usepackage{listings}
\usepackage{pgf-umlcd}
\usepackage{blindtext}
\usepackage{enumitem}
%\usepackage{lmodern}   % Optional fonts. (not necessary)
%\usepackage{tabularx}
%\usepackage{microtype} % Provides some typographic improvements over default settings
%\usepackage{placeins}  % For aligning images with \FloatBarrier
%\usepackage{booktabs}  % For nice-looking tables
%\usepackage{titlesec}  % More granular control of sections.

% DOCUMENT INFO
% =============
\department{Department of Computing Science}
\coursename{Parallel Programming 7.5 p}
\coursecode{5DV152}
\title{Exercises, Chapter/Topic 1}
\author{Lorenz Gerber ({\tt{dv15lgr@cs.umu.se}} {\tt{lozger03@student.umu.se}})}
\date{2017-01-26}
%\revisiondate{2016-01-18}
\instructor{Lars Karlsson / Mikael Ränner}


% DOCUMENT SETTINGS
% =================
\bibliographystyle{plain}
%\bibliographystyle{ieee}
\pagestyle{fancy}
\raggedbottom
\setcounter{secnumdepth}{2}
\setcounter{tocdepth}{2}
%\graphicspath{{images/}}   %Path for images

\usepackage{float}
\floatstyle{ruled}
\newfloat{listing}{thp}{lop}
\floatname{listing}{Listing}



% DEFINES
% =======
%\newcommand{\mycommand}{<latex code>}

% DOCUMENT
% ========
\begin{document}
\lstset{language=C}
\maketitle
\thispagestyle{empty}
\newpage
\tableofcontents
\thispagestyle{empty}
\newpage

\clearpage
\pagenumbering{arabic}

\section{Introduction}
This report is part of the mandatory coursework. It describes the solutions for several chosen exercises from the course book \cite{pacheco2011}.
\section{4.1 - Generalization of matrix-vector multiplication}
If we keep the same scheme of parallelization as mentioned in the book (outer for...loop), generalzation can be implemented rather easy, bascially in the same way as already shown in exercise 1.1:

\begin{verbatim}                                                                                             
my_first_i = k * m / p + ( k < m mod p ? k : m mod p )                                                       
my_last_i = (k + 1) * m / p + (k + 1 < m mod p ? k + 1 : m mod p)  
\end{verbatim}

It is not useful to parallelize into n as one thread needs to process as this would create a mutex for acces to shared variables. 

\section{4.2 - Physical data distribution}


\section{4.8 - Deadlock}

\section{4.11 - Linked list troubles}

\section{4.12 - Linked list insert and delete with read-write lock}

\section{4.17 - False sharing}

\section{A4.1 - Histogram}

\section{A4.3 - Trapezoidal rule}

\section{A4.4 - Fork/join overhead}

\section{A4.5 - Task queue} 



\addcontentsline{toc}{section}{\refname}
\bibliography{references}

\end{document}
