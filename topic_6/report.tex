\documentclass[a4paper,11pt,twoside]{article}
%\documentclass[a4paper,11pt,twoside,se]{article}

\usepackage{UmUStudentReport}
\usepackage{verbatim}   % Multi-line comments using \begin{comment}
\usepackage{courier}    % Nicer fonts are used. (not necessary)
\usepackage{pslatex}    % Also nicer fonts. (not necessary)
\usepackage[pdftex]{graphicx}   % allows including pdf figures
\usepackage{listings}
\usepackage{pgf-umlcd}
\usepackage{blindtext}
\usepackage{enumitem}
%\usepackage{lmodern}   % Optional fonts. (not necessary)
%\usepackage{tabularx}
%\usepackage{microtype} % Provides some typographic improvements over default settings
%\usepackage{placeins}  % For aligning images with \FloatBarrier
%\usepackage{booktabs}  % For nice-looking tables
%\usepackage{titlesec}  % More granular control of sections.

% DOCUMENT INFO
% =============
\department{Department of Computing Science}
\coursename{Parallel Programming 7.5 p}
\coursecode{5DV152}
\title{Exercises, Chapter/Topic 6}
\author{Lorenz Gerber ({\tt{dv15lgr@cs.umu.se}} {\tt{lozger03@student.umu.se}})}
\date{2017-03-09}
%\revisiondate{2016-01-18}
\instructor{Lars Karlsson / Mikael Ränner}


% DOCUMENT SETTINGS
% =================
\bibliographystyle{plain}
%\bibliographystyle{ieee}
\pagestyle{fancy}
\raggedbottom
\setcounter{secnumdepth}{2}
\setcounter{tocdepth}{2}
%\graphicspath{{images/}}   %Path for images

\usepackage{float}
\floatstyle{ruled}
\newfloat{listing}{thp}{lop}
\floatname{listing}{Listing}



% DEFINES
% =======
%\newcommand{\mycommand}{<latex code>}

% DOCUMENT
% ========
\begin{document}
\lstset{language=C}
\maketitle
\thispagestyle{empty}
\newpage
\tableofcontents
\thispagestyle{empty}
\newpage

\clearpage
\pagenumbering{arabic}

\section{Introduction}
This report is part of the mandatory coursework. It describes the solutions for several chosen exercises from the course book \cite{pacheco2011}.

\section{Modification of $n$ body solver}
Yes it would be possible to remove the the inner \verb+for+ loop. However, as the calculation of force makes use of the position of all particles, updates to the position would need to be stored in a temporary variable and updated after calculating the force for all particles of the current step. This would probably deteriorate a potential speed gain from removing some structure again.

\section{Extrapolation of execution time for $n$ body solver}
The serial n-body solver was run according to the specifications in the book for 500 to 2000 particles with 3 replicates. The timings were then plotted as shown in figure \ref{fig:nbody}. From the plot shape, a transformation was guessed ($x^2$). Then a linear model was fitted and extrapolated to 24h. The approximation for 24h was 70´000 particles.

\section{Eliminating implied barriers}
Eliminating the implied barriers in the basic OpenMP implementation has a similar consequence as described in question 6.1. As calculation of the force on particles needs the position information of each and every particle, threads that `run' ahead to the position update block and change the positions while others still calculate the force would result in wrong force values and as a consequence also false positions. Hence, removing the implied barriers would result in a wrong and unpredictable result.
  
\section{DAXPY}

\section{L2 cache misses}

\section{Local/global index conversions}

\section{Stack splitting in TSP}

\section{Choosing an API}



\addcontentsline{toc}{section}{\refname}
\bibliography{references}

\end{document}
